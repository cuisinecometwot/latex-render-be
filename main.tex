\documentclass{article} % Specifies the document class

\usepackage[utf8]{inputenc} % Encoding of the document
\usepackage{amsmath} % Package for advanced math typesetting
\usepackage{graphicx} % Package for including images

\title{Sample LaTeX Document} % Title of the document
\author{John Doe} % Author of the document
\date{\today} % Date of the document

\begin{document} % Start of the document

\maketitle % Generates the title

\section{Introduction} % Section heading
This is a simple example of a LaTeX document. LaTeX is widely used for typesetting documents, especially those that include mathematical content.

\section{Mathematics} % Another section
Here is an example of a mathematical equation:

\begin{equation} % Equation environment
E = mc^2
\end{equation}

\section{Lists} % Section for lists
You can create lists in LaTeX:

\subsection{Unordered List} % Subsection
\begin{itemize}
    \item Item 1
    \item Item 2
    \item Item 3
\end{itemize}

\subsection{Ordered List} % Subsection
\begin{enumerate}
    \item First item
    \item Second item
    \item Third item
\end{enumerate}

\section{Tables} % Section for tables
Here is an example of a table:

\begin{table}[h]
    \centering
    \begin{tabular}{|c|c|c|}
        \hline
        Header 1 & Header 2 & Header 3 \\ \hline
        Row 1    & Data 1   & Data 2    \\ \hline
        Row 2    & Data 3   & Data 4    \\ \hline
    \end{tabular}
    \caption{Sample Table} % Caption for the table
    \label{tab:sample_table} % Label for referencing the table
\end{table}

\section{Conclusion} % Conclusion section
This document provides a brief overview of some of the features of LaTeX. You can use it to create professional-looking documents with ease.

\end{document} % End of the document
